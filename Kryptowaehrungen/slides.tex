\documentclass[aspectratio=169]{beamer}

% Packages
\usepackage{hyperref}

% Colors
\usepackage{xcolor}
\definecolor{maincolor}{RGB}{102, 0, 51}
\definecolor{secondarycolor}{RGB}{153, 0, 153}
\definecolor{thirdcolor}{RGB}{102, 0, 204}

% Beamer theme
\setbeamercolor{titlelike}{fg=maincolor}
\setbeamercolor{frametitle}{fg=maincolor}
\setbeamertemplate{frametitle}{
    \vspace{0,4cm}\LARGE\textbf{\insertframetitle}
    \hfill
    \normalsize\insertsection
}
\setbeamertemplate{navigation symbols}{}
\setbeamertemplate{footline}{
    \;\;\;\;
    \textcolor{secondarycolor}{\textbf{\insertframenumber/\inserttotalframenumber}}
    \hfill
    \href{https://github.com/claussmann}{\textcolor{secondarycolor}{@claussmann}}
    \;\;\;\;\vspace{0.2cm}
}
\setbeamertemplate{itemize item}{\color{maincolor}$\blacktriangleright$}
\setbeamertemplate{itemize subitem}{\color{secondarycolor}$\blacktriangleright$}

\title{\textbf{\Huge Krypto Währungen}}
\author{\textcolor{secondarycolor}{\textbf{Christian Laussmann}}}
\date{}

\begin{document}

\frame{\titlepage}





\section{Einführung}


\begin{frame}{Bargeldzahlung}
    Bild: Anna gibt Bert eine Münze.

    \pause
    \textbf{Das funktioniert, weil:}
    \begin{itemize}
        \item Anna \textbf{besitzt} die Münze und \textbf{darf} sie abgeben.
        \item Bert \textbf{besitzt} anschließend die Münze und kann damit bezahlen.
        \item Anna kann Bert die Münze nicht \textbf{wegnehmen} oder die Münze \textbf{erneut ausgeben}.
    \end{itemize}
\end{frame}



\begin{frame}{Überweisung}
    Bild: Anna sitzt am Computer beim Onlinebanking. Ihre Bank redet mit der Bank von Bert. Bert erhält eine Nachricht aufs Handy, dass das Geld auf seinem Konto ist.

    \pause
    \begin{itemize}
        \item \textbf{Gültigkeit:} Annas Bank prüft Annas Kontostand. Gesetze und Zentralbanken verhindern Betrug zwischen den Banken.
        \item \textbf{Authorisierung:} Anna hat Zugangsdaten zum Onlinebanking.
        \item \textbf{Unveränderlichkeit:} Sichergestellt durch Gesetze und Zentralbanken.
    \end{itemize}
\end{frame}



\begin{frame}{Krypto-Überweisung}
    Bild: Anna sitzt am Computer am Krypto-Wallet. Sie schreibt in eine Datenbank im Internet den Satz \emph{Anna überweist 5 coins an Bert} Bert schaut in die Datenbank.

    \pause
    \begin{itemize}
        \item \textbf{Gültigkeit:} Wer prüft, ob Anna genug Geld hat?
        \item \textbf{Authorisierung:} Wie ist garantiert, dass jeder nur eigenes Geld überweisen kann?
        \item \textbf{Unveränderlichkeit:} Wer garantiert, dass Transaktionen nicht rückgängig gemacht werden?
    \end{itemize}

    \vfill
    \begin{itemize}
        \item \textbf{Niemand vertraut den anderen.} Vertrauen in die Währung muss anders hergestellt werden.
    \end{itemize}
\end{frame}


\section{Unveränderlichkeit}

\begin{frame}{Die Blockchain}
    \begin{itemize}
        \item Transaktionen zu Blöcken gebündelt
        \item Jeder Block ist mit Vorgänger verbunden.
        \pause
        \textbf{Aber wie?}
    \end{itemize}
\end{frame}


\begin{frame}{Hash-Pointer}
    \begin{itemize}
        \item Jeder Block enthält Hash des Vorgängers
        \item Vorgänger und Reihenfolge eindeutlig identifiziert
    \end{itemize}
\end{frame}


\begin{frame}{Gebrochene Kette}
    \begin{itemize}
        \item Ändern eines Blocks, ändert seinen Hash $\Rightarrow$ `Kette bricht'.
    \end{itemize}
\end{frame}


\begin{frame}{Reparatur einer gebrochenen Kette}
    \begin{itemize}
        \item Hash-Pointer muss in \textbf{allen} folgenden Blöcken sngepasst werden
        \pause
        \item \textbf{Wir wollen das natürlich verhindern!}
    \end{itemize}
\end{frame}


\begin{frame}{Nonce und Hash-Challenge}
    \begin{itemize}
        \item Akzeptieren nur Hash-Werte mit mindestens $X$ Nullen
        \item Passende Nonce durch ausprobieren finden \textbf{(sogenanntes /emph{Mining})}
        \item Angreifer müssten schneller alte Blöcke bearbeiten, als Miner neue erstellen.
    \end{itemize}
\end{frame}






\section{Authorisierung}

\begin{frame}{Accounterstellung}
		\begin{itemize}
				\item Wallet erzeugt Public- und Private-Key
				\item Public-Key (bzw. Ableitung davon) bildet `Kontonummer'
		\end{itemize}
\end{frame}


\begin{frame}{Überweisung Authorisieren}
		\begin{itemize}
				\item Signatur der Überweisung mittels Private-Key
				\item Signatur kann durch Public-Key von jedem überprüft werden
		\end{itemize}
\end{frame}





\section{Gültigkeit}

\begin{frame}{Wann ist Transaktion gültig?}
		\begin{itemize}
				\item Transaktion wurde authorisiert $\Rightarrow$ Signatur
				\item Sender hat genug Geld $\Rightarrow$ In vorherigen Blöcken nach unverbrauchter Gutschrift suchen
		\end{itemize}
\end{frame}





\section{Mining}

\begin{frame}{Blöcke erschaffen}
		\begin{itemize}
				\item Miner haben jeweils Kopie der Blockchain
				\item Miner versuchen jeweils neue Blöcke anzuhängen und allen mitzuteilen
				\item Nur gültige neue Blöcke werden von anderen Minern aufgenommen
		\end{itemize}
\end{frame}


\begin{frame}{Woher kommt das Geld?}
		\begin{itemize}
				\item Miner bekommt Belohnung für neuen Block
				\item Miner werden \textbf{aus Eigennutz das Protokoll einhalten}, da ihr Block sonst nicht akzeptiert wird
		\end{itemize}
\end{frame}


\begin{frame}{Fork}
		\begin{itemize}
			\item Mittelfristig gewinnt die längere Kette
		\end{itemize}
\end{frame}


\end{document}