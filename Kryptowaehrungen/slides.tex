\documentclass[aspectratio=169]{beamer}

% Packages
\usepackage{hyperref}

% Colors
\usepackage{xcolor}
\definecolor{maincolor}{RGB}{102, 0, 51}
\definecolor{secondarycolor}{RGB}{153, 0, 153}
\definecolor{thirdcolor}{RGB}{102, 0, 204}

% Beamer theme
\setbeamercolor{titlelike}{fg=maincolor}
\setbeamercolor{frametitle}{fg=maincolor}
\setbeamertemplate{frametitle}{\vspace{0,4cm}\huge\bfseries\insertframetitle}
\setbeamertemplate{navigation symbols}{}
\setbeamertemplate{footline}{
    \;\;\;\;
    \textcolor{secondarycolor}{\textbf{\insertpagenumber/\inserttotalframenumber}}
    \hfill
    \href{https://github.com/claussmann}{\textcolor{secondarycolor}{@claussmann}}
    \;\;\;\;\vspace{0.2cm}
}

\title{\textbf{\Huge Krypto Währungen}}
\author{\textcolor{secondarycolor}{\textbf{Christian Laussmann}}}
\date{}

\begin{document}

\frame{\titlepage}



\begin{frame}{Bargeldzahlung}
    Bild: Anna gibt Bert eine Münze.

    \pause
    \textbf{Das funktioniert, weil:}
    \begin{itemize}
        \item Anna \textbf{besitzt} die Münze und \textbf{darf} sie abgeben.
        \item Bert \textbf{besitzt} anschließend die Münze und kann damit bezahlen.
        \item Anna kann Bert die Münze nicht \textbf{wegnehmen} oder die Münze \textbf{erneut ausgeben}.
    \end{itemize}
\end{frame}



\begin{frame}{Überweisung}
    Bild: Anna sitzt am Computer beim Onlinebanking. Ihre Bank redet mit der Bank von Bert. Bert erhält eine Nachricht aufs Handy, dass das Geld auf seinem Konto ist.

    \pause
    \begin{itemize}
        \item \textbf{Gültigkeit:} Annas Bank prüft Annas Kontostand. Gesetze und Zentralbanken verhindern Betrug zwischen den Banken.
        \item \textbf{Authorisierung:} Anna hat Zugangsdaten zum Onlinebanking.
        \item \textbf{Unveränderlichkeit:} Sichergestellt durch Gesetze und Zentralbanken.
    \end{itemize}
\end{frame}



\begin{frame}{Krypto-Überweisung}
    Bild: Anna sitzt am Computer am Krypto-Wallet. Sie schreibt in eine Datenbank im Internet den Satz \emph{Anna überweist 5 coins an Bert} Bert schaut in die Datenbank.

    \pause
    \begin{itemize}
        \item \textbf{Gültigkeit:} Wer prüft, ob Anna genug Geld hat?
        \item \textbf{Authorisierung:} Wie ist garantiert, dass jeder nur eigenes Geld überweisen kann?
        \item \textbf{Unveränderlichkeit:} Wer garantiert, dass Transaktionen nicht rückgängig gemacht werden?
    \end{itemize}

    \vfill
    \begin{itemize}
        \item \textbf{Niemand vertraut den anderen.} Vertrauen in die Währung muss anders hergestellt werden.
    \end{itemize}
\end{frame}


\end{document}